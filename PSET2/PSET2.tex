%! Author = Kevin Lin
%! Date = 4/1/2025

% Preamble
\documentclass[11pt,a4paper,margin=1in]{article}

% Packages
\usepackage{amsmath}

\title{PSET 2}
\author{Kevin Lin}
\date{4/1/2025}

% Document
\begin{document}
\maketitle
\section{}
\begin{flushleft}
    Assume that the average tree absorbs 10 kg of CO$_2$ per year, and 
    that 1000 trees are planted per hectare. 1 hectare is 0.01 km$^2$. Thus, the mass
    of CO$_2$ absorbed per year per square kilometer is:
\end{flushleft}
\begin{gather*}
    10 \frac{\text{kg}}{\text{tree}} \times 10^3 
        \frac{\text{trees}}{\text{hectare}} \times 10^2 
        \frac{\text{hectares}}{\text{km}^2} = 10^6 
        \frac{\text{kg}}{\text{km}^2}\\
    10^6 \frac{\text{kg}}{\text{km}^2} 
        = 10^6\times10^3 \frac{\text{g}}{\text{km}^2}
        = 10^9 \frac{\text{g}}{\text{km}^2} \text{ per year}\\
\end{gather*}

\section{}
\begin{flushleft}
    Assume we lose 10 million hectares of forest each year, and that deforestation 
    started 12,000 years ago. Also assume that the amount of CO$_2$ lost from a
    square kilometer of trees yearly is the same amount that can be stored 
    (our answer from 1). We know 1 hectare is 0.01 km$^2$, so this means we lose:
\end{flushleft}
\begin{gather*}
    10^7 \frac{\text{hectares}}{\text{year}} \times 10^{-2} 
        \frac{\text{km}^2}{\text{hectare}} = 10^5 \frac{\text{km}^2 
            \text{ (of trees)}}{\text{year}}\\
    10^5 \frac{\text{km}^2 \text{ (of trees)}}{\text{year}} = 
        10^5 \text{ km}^2 \times 10^9 \frac{\text{g}}{\text{km}^2 \cdot \text{year}} 
        = 10^{14} \frac{\text{g (of CO}_2)}{\text{year}}\\
    12000 = 1.2 \times 10^4 \approx 10^4 \text{ years}\\
    10^4 \text{ years} \times 10^{14} \frac{\text{g}}{\text{year}} = 10^{18} 
        \text{ g of CO}_2 \text { total}
\end{gather*}
\begin{flushleft}
    From class, we know that the total mass of CO$_2$ emitted by cars in the US
    per year is around 10$^{15}$g. This means that it would only take around 10$^3$ 
    years of car emissions to equal the total amount of CO$_2$ emitted from deforestation.
    Note that this is a very rough estimation, as the deforestation rate is definitely
    not constant over the last 12,000 years, but more likely exponential. However,
    the relative order of magnitude is still the same. This still highlights 1),
    the amount of deforestation that has occurred, and 2), the amount of CO$_2$
    that comes from driving emissions in the US alone. 
\end{flushleft}
\end{document}