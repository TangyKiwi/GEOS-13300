%! Author = Kevin Lin
%! Date = 5/9/2025

% Preamble
\documentclass[11pt,a4paper,margin=1in]{article}

% Packages
\usepackage{amsmath}
\usepackage{enumerate}
\usepackage{cancel}

\title{PSET 5}
\author{Kevin Lin}
\date{5/9/2025}

\begin{document}
\maketitle
\section{}
\begin{enumerate}[A.]
    \item 
    We are looking for areas in the plot that feature tightly packed contours,
    as this indicates a steep gradient. A steep gradient also indicates high speeds
    for our jet stream.
    \item 
    On average, the jet stream flows westerly. Looking at the H and the L directly 
    to its left in the plot, we know that $d\psi$ flows strictly in the $-y$ direction, 
    thus, $u = +y$ and $v = 0$, indicating a flow from the ``bottom'' of the plot
    to the ``top''. Because the plot is a polar projection oriented with the North 
    Pole at the center, this means the jet stream is flowing from west to east. 
    \item 
    Over the western United States, the jet stream is meandering southward, as
    the contours are sloping downwards. Thus, temperatures would be colder as 
    the jet stream brings in colder air from the north.
    \item 
    Over the eastern United States, the jet stream is meandering northward, as
    the contours are sloping upwards. Thus, temperatures would be warmer as
    the jet stream brings in warmer air from the south.
    \item 
    From our analysis in B, we know circulation around the lows is counterclockwise.
    \item 
    From our analysis in B, we know circulation around the highs is clockwise.
\end{enumerate}
\end{document}