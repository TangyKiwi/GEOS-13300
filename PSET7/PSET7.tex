%! Author = Kevin Lin
%! Date = 5/9/2025

% Preamble
\documentclass[11pt,a4paper,margin=1in]{article}

% Packages
\usepackage{amsmath}
\usepackage{enumerate}
\usepackage{cancel}

\title{PSET 5}
\author{Kevin Lin}
\date{5/9/2025}

\begin{document}
\maketitle
\section{}
\begin{enumerate}[A.]
    \item 
    Near the surface at the center of the tropical cyclone, the winds spiral 
    inward, thus the divergence of the horizontal wind field is negative as the 
    winds are converging.
    \item 
    At the center and top of the tropical cyclone, the winds spiral outward,
    thus the divergence of the horizontal wind field is positive as the winds are
    diverging.
    \item 
    From the surface center of the cyclone, we know the winds are converging, and
    thus is drawing additional air mass into the cyclone. This air mass is then
    transported upward by the vertically convecting eye wall, and consequently must
    "leave" the cyclone and diverges at the top. Thus, the sign of divergence at
    the surface must be opposite to the sign at the top of the cyclone.
    \item 
    \begin{gather*}
        \vec{\nabla} \cdot \vec{v} = \frac{\partial u}{\partial x} + 
            \frac{\partial v}{\partial y} \\
        \vec{\nabla} \cdot \vec{v} \sim \frac{m/s}{m} + \frac{m/s}{m} = \frac{1}{s}
    \end{gather*}
    \item 
    Let the radial speed wind in the eyewall be $V_r = -1 \text{ ms}^{-1}$
    (negative because converging). Because the diameter of the cyclone is 50 km, 
    the radius of the cyclone is  25 km = 25000 m. The radial speed wind can be 
    expressed in Cartesian vector form as:
    \begin{gather*}
        \vec{V_r} = V_r \cos(\theta) \hat{i} + V_r \sin(\theta) \hat{j} \\
    \end{gather*}
    From polar coordinates, we know that:
    \begin{gather*}
        \cos(\theta) = \frac{x}{r}, \  \sin(\theta) = \frac{y}{r} \\
    \end{gather*}
    where $r = \sqrt{x^2 + y^2}$. Thus, we can express the radial speed wind
    in Cartesian coordinates as:
    \begin{gather*}
        u = V_r \cos(\theta) = V_r \frac{x}{r} = -\frac{x}{r} \\
        v = V_r \sin(\theta) = V_r \frac{y}{r} = -\frac{y}{r}
    \end{gather*}
    Thus, the divergence is:
    \begin{gather*}
        \vec{\nabla} \cdot \vec{v} = \frac{\partial u}{\partial x}\left(-\frac{x}{r}\right) 
            + \frac{\partial v}{\partial y}\left(-\frac{y}{r}\right) \\
        \frac{\partial u}{\partial x}\left(-\frac{x}{r}\right) = -\frac{\partial u}
            {\partial x}\left(\frac{x}{\sqrt{x^2 + y^2}}\right) = -\left(\dfrac{
                r - x \frac{2x}{2r}}{r^2}\right) = -\left(\frac{1}{r} - 
                \frac{x^2}{r^3}\right)\\
        \frac{\partial v}{\partial y}\left(-\frac{y}{r}\right) = -\left(\frac{1}{r}
            - \frac{y^2}{r^3}\right) \\
        \vec{\nabla} \cdot \vec{v} = -\left(\frac{1}{r} - \frac{x^2}{r^3}\right)
            - \left(\frac{1}{r} - \frac{y^2}{r^3}\right) \\
        \vec{\nabla} \cdot \vec{v} = -\left(\frac{2}{r} - \frac{x^2 + y^2}{r^3}\right) \\
        \vec{\nabla} \cdot \vec{v} = -\left(\frac{2}{r} - \frac{r^2}{r^3}\right) 
            = -\frac{1}{r} = -\frac{1 \text{ ms}^{-1}}{25000 \text{m}} = -4 
            \times 10^{-5} \text{ s}^{-1}
    \end{gather*}
    \item 
    We know that northern hemisphere cyclones spin counterclockwise, thus the
    vorticity at the center of the cyclone is positive.
    \item 
    \begin{gather*}
        \vec{\nabla} \times \vec{v} = \frac{\partial v}{\partial x} - 
            \frac{\partial u}{\partial y} \\
        \vec{\nabla} \times \vec{v} \sim \frac{m/s}{m} - \frac{m/s}{m} = \frac{1}{s}
    \end{gather*}
    \item 
    The cyclonic wind speed is tangential to the rotating cyclone and thus 
    perpendicular to the radial wind speed. Following our calculations from E, 
    we can modify our Cartesian coordinates:
    \begin{gather*}
        u = -V_\theta \sin(\theta) = -V_\theta \frac{y}{r} \\
        v = V_\theta \cos(\theta) = V_\theta \frac{x}{r}
    \end{gather*}
    We make $u$ negative in order to maintain the counterclockwise rotation of 
    the cyclone. Additionally, tangential velocity scales with the radius, so 
    adjusting $V_\theta$ accordingly, we get:
    \begin{gather*}
        u = -\frac{V_\theta r}{R} \frac{y}{r} = -\frac{V_\theta}{R} y \\
        v = \frac{V_\theta r}{R} \frac{x}{r} = \frac{V_\theta}{R} x
    \end{gather*}
    Thus, our vorticity is:
    \begin{gather*}
        \vec{\nabla} \times \vec{v} = \frac{\partial v}{\partial x}\left(
            \frac{V_\theta}{R}x\right) - \frac{\partial u}{\partial y}\left(
            -\frac{V_\theta}{R}y\right) \\
        \vec{\nabla} \times \vec{v} = \frac{V_\theta}{R} + \frac{V_\theta}{R} \\
        \vec{\nabla} \times \vec{v} = \frac{2V_\theta}{R} = \frac{2 \times 60 
            \text{ ms}^{-1}}{25000 \text{ m}} = 4.8 \times 10^{-3} \text{ s}^{-1} 
    \end{gather*}
    \item 
    If the hurricane is at a latitude of 20$^\circ$ N, and $\Omega = 7.3 \times
    10^{-5} \text{ s}^{-1}$, we can calculate the planetary vorticity:
    \begin{gather*}
        f = 2 \times 7.3 \times 10^{-5} \text{ s}^{-1} \sin(20^\circ) \\
        f = 5.13 \times 10^{-5} \text{ s}^{-1}
    \end{gather*}
    This makes the vorticity of the cyclone about 93 times stronger than the
    vorticity of the planet.
    \item 
    \begin{gather*}
        u = -\frac{\partial \psi}{\partial y} = -\frac{\partial}{\partial y} 
            \left(2u_0r_0^{\frac{1}{2}}(x^2 + y^2)^{\frac{1}{4}}\right)\\
        u = -2u_0r_0^{\frac{1}{2}}\left(\frac{1}{4}(x^2 + y^2)^{-\frac{3}{4}}(2y)
            \right) = -\dfrac{u_0r_0^{\frac{1}{2}}y}{(x^2 + y^2)^{\frac{3}{4}}}\\
        v = \frac{\partial \psi}{\partial x} = \dfrac{u_0r_0^{\frac{1}{2}}x}
            {(x^2 + y^2)^{\frac{3}{4}}}
    \end{gather*}
    \item 
    Note that $x^2 + y^2 = r^2\cos^2(\theta) + r^2\sin^2(\theta) = r^2$, so:
    \begin{gather*}
        u = -\frac{u_0r_0^{\frac{1}{2}}r\sin(\theta)}{(r^2)^{\frac{3}{4}}} = 
            -\frac{u_0r_0^{\frac{1}{2}}\sin(\theta)}{r^{\frac{1}{2}}}\\
        v = \frac{u_0r_0^{\frac{1}{2}}r\cos(\theta)}{(r^2)^{\frac{3}{4}}} =
            \frac{u_0r_0^{\frac{1}{2}}\cos(\theta)}{r^{\frac{1}{2}}}
    \end{gather*} 
    \item 
        The velocities decay by a factor of $r^{-\frac{1}{2}}$ as the distance 
        from the center of the cyclone increases. 
\end{enumerate}
\end{document}