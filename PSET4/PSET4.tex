%! Author = Kevin Lin
%! Date = 4/18/2025

% Preamble
\documentclass[11pt,a4paper,margin=1in]{article}

% Packages
\usepackage{amsmath}
\usepackage{enumerate}

\title{PSET 4}
\author{Kevin Lin}
\date{4/18/2025}

\begin{document}
\maketitle
\section{}
We know that the relative humidity is defined as:
\begin{gather*}
    RH = \frac{e}{e_s} \cdot 100\%
\end{gather*}
Where $e$ is the actual vapor pressure and $e_s$ is the saturation vapor pressure.
From the graph, we can see that at points B and C, the vapor pressure is around
5 mb. The vapor pressure of human breath is around 55 mb. When the two air parcels 
mix, the vapor pressure and temperature at that state can be taken as their average.
The saturation vapor pressure is then calculated at the average temperature. 

Thus, at point B, the average vapor pressure is $(55 + 5) / 2 = 30$ mb and the 
temperature is $(35 + 20) / 2 = 27.5$ C$^{\circ}$. The saturation vapor pressure 
at this temperature is around 40 mb. Thus, the relative humidity at point B is:
\begin{gather*}
    RH = \frac{30 \text{ mb}}{40 \text{ mb}} \cdot 100\% = 75\%
\end{gather*}

At point C, the average vapor pressure is also $(55 + 5) / 2 = 30$ mb and the
temperature is $(35 + 5) / 2 = 20$ C$^{\circ}$. The saturation vapor pressure at
this temperature is around 25 mb. Thus, the relative humidity at point C is:
\begin{gather*}
    RH = \frac{30 \text{ mb}}{25 \text{ mb}} \cdot 100\% = 120\%
\end{gather*}
This means that the air parcel at point C is supersaturated, which means that
the air parcel is unstable and will condense into liquid water. Thus, at point C,
we expect to see our breath, but not at point B.

\section{}
We know that this is an adiabatic process. Let the initial temperature and pressure
be $T_1$ and $P_1$, representing the temperature and pressure of the air outside
the jet. Let the temperature and pressure of the air inside the jet be $T_2$ 
and $P_2$ after it enters the aircraft. Thus, from an adiabatic process, we know 
that:
\begin{gather*}
    T_1 \left( \frac{p_0}{P_1} \right)^{\frac{R}{c_p}} = T_2 \left( \frac{p_0}{P_2} 
    \right)^{\frac{R}{c_p}}\\
    T_2 = T_1 \left( \frac{P_2}{P_1} \right)^{\frac{R}{c_p}}\\
\end{gather*}
Assume that $T_1$ is 220 K and $P_1$ is 1000 mb = $10^3$ mb. The pressure outside 
the jet $P_2$ 300 mb $\approx \pi \times 10^2$ mb. We know $R$ is 287 $\approx 
\pi \times 10^2$ J K$^{-1}$ Kg$^{-1}$, and that $c_p \approx 10^3$ K$^{-1}$ 
Kg$^{-1}$. Thus, we get $T_2$ as:
\begin{gather*}
    T_2 = 220 \text{ K} \left( \frac{10^3 \text{ mb}}
        {\pi \times 10^2 \text{ mb}} \right)^{\frac{\pi \times 10^2 \text{J K}^{-1} 
        \text{ Kg}^{-1}}{10^3 \text{J K}^{-1}\text{ Kg}^{-1}}}\\
    T_2 = 220 \text{ K} \pi^{\frac{\pi}{10}}\\
    T_2 \approx 220 \text{ K} \pi^{\frac{1}{\pi}} \approx \pi \times 10^2 \text{ K}\\
\end{gather*}
Converting this back to C$^{\circ}$ gives us $\approx$ 41 C$^{\circ}$. This is
about a 20 C$^{\circ}$ increase in temperature. Thus, the air inside the jet needs
to be cooled down by 20 C$^{\circ}$ to reach an appropriate temperature for the
cabin (assume 20 C$^{\circ}$ for room temperature). 
\end{document}