%! Author = Kevin Lin
%! Date = 3/26/2025

% Preamble
\documentclass[11pt,a4paper,margin=1in]{article}

% Packages
\usepackage{amsmath}

\title{PSET 1}
\author{Kevin Lin}
\date{3/26/2025}

% Document
\begin{document}
\maketitle
\section{}
\begin{flushleft}
    $0.039\% = 0.00039 = 3.9\times10^{-4} \approx \pi\times10^{-4} = $ fraction of
        $\text{CO}_2$ molecules\\
    From class, we estimate the atmosphere's mass as $\pi\times10^{21}$ g\\
    Thus, we have $\pi\times10^{-4} \times \pi\times10^{21} = \pi^2\times10^{17}$ g
        of $\text{CO}_2$\\
    Using $\pi^2 \approx 10$, we then have $10^{18}$ g of $\text{CO}_2$ in the
        atmosphere\\
    We estimate the density of dry ice to be $1564 \frac{\text{kg}}{\text{m}^3}
        \approx 10^3 \frac{\text{kg}}{\text{m}^3} = 10^6 \frac{\text{g}}{\text{m}^3}$\\
    This then gives us $\dfrac{10^{18} \text{g}}{10^6 \frac{\text{g}}{\text{m}^3}}
        = 10^{12} \text{ m}^3$ of $\text{CO}_2$ as dry ice\\
    From class, we know estimate the Earth's surface area as $\pi\times10^{14}\text{ m}^2$\\
    Thus, to find the snowfall depth, we calculate V/SA:\\
    $\dfrac{10^{12}\text{ m}^3}{\pi\times10^{14}\text{ m}^2} = \dfrac{10}{\pi}\cdot
        \dfrac{10^{11}}{10^{14}}\cdot\dfrac{\text{ m}^3}{\text{ m}^2}$\\
    Using $\frac{10}{\pi} \approx \pi$ we finally get $\pi\times10^{-3}\text{ m}$
        of snowfall depth.
\end{flushleft}

\section{}
\begin{flushleft}
    We estimate $\text{O}_2$ to be $21\% \approx 20\%$ of the Earth's atmosphere.\\
    From class, we estimate the atmosphere's mass as $\pi\times10^{21}$ g.\\
    Thus, we have $2\times10^{-1}\times\pi\times10^{21} = 2\pi\times10^{20}$ g
        of $\text{O}_2$ in the atmosphere, and $10\%$ of such would be
        $2\pi\times10^{19}$ g.\\
    From class, we estimate the average human breath to be $1\text{ L} = 10^{-3}
        \text{ m}^{3}$, with a mass of $1$ g.\\
    Assuming a human breathes a normal consistency of air and not pure $\text{O}_2$,
        a single human breath would also be $20\%$ $\text{O}_2$, which would be
        $2\times10^{-1}$ g of $\text{O}_2$.\\
    Thus, it would take $\dfrac{2\pi\times10^{19}}{2\times10^{-1}} = \pi\times10^{20}$
        breaths for a single human to use up $10\%$ of all atmospheric $\text{O}_2$.\\
    We estimate the average human breath to take $2-3$ sec $\approx \pi$ sec.\\
    Thus, $\pi\times10^{20}$ breaths would take $\pi\times\pi\times10^{20}$ sec.\\
    Using $\pi^2 \approx 10$, we finally get $10^{21}$ seconds.
\end{flushleft}
\end{document}