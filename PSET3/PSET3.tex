%! Author = Kevin Lin
%! Date = 4/11/2025

% Preamble
\documentclass[11pt,a4paper,margin=1in]{article}

% Packages
\usepackage{amsmath}
\usepackage{enumerate}

\title{PSET 3}
\author{Kevin Lin}
\date{4/11/2025}

\begin{document}
\maketitle
\section{}
From class we know the hydrostatic balance equation:
\[
    \frac{dP}{dz} = -\rho g
\]
Let the pressure at the surface (also our lower boundary) be $P$. Because the
atmosphere is homogenous, we know $\rho$ is constant. Thus, reorganizing our
equation and integrating from the surface (0, $P_0$) to some height ($H$) (where
$P = 0$) gives us:
\begin{gather*}
    \int_{P_0}^{0} dP = -\rho g \int_{0}^{H}  dz\\
    0 - P_0 = -\rho g (H - 0)\\
    -P_0 = -\rho g H\\
    H = \frac{P_0}{\rho g}
\end{gather*}
At the surface, $P_0 = \rho R T_0$, where $T_0$ is our lower boundary temperature.
Thus:
\begin{gather*}
    H = \frac{\rho R T_0}{\rho g}\\
    H = \frac{R T_0}{g}
\end{gather*}
Therefore, $H$, which also represents the scale height of the atmosphere, must
be a finite value as $R$ and $g$ are both constants, and $T_0$ is a constant
determined at the lower boundary (surface), which is definitely not infinity.

\section{}
\begin{enumerate}[A.]
\item
    Recall that the tropopause is the boundary between the troposphere and 
    stratosphere where the temperature changes from decreasing to increasing
    with height. We also know that pressure decreases with height. Combining
    these two facts, we can conclude that the troposphere is represented by the
    dotted line in the figure. We know that the troposphere is around 10 km in
    height, and that from the figure, its temperature changes from 300K to 200K.
    Thus, its lapse rate is:
    \begin{gather*}
        \Gamma = -\frac{dT}{dz} = -\frac{200\text{ K} - 300 \text{ K}}
            {10\text{ km}} = 10 \text{ K km}^{-1}
    \end{gather*}
    This is the same as the dry adiabatic lapse rate.
\item
    The 700mbar to 300mbar layer is represented by the solid line in the figure,
    which is roughly linear. The average temperature of this layer is around 325K.
    Thus, the pressure scale height is:
    \begin{gather*}
        H = \frac{RT_0}{g}\\
        H = \frac{8.3 \text{J mol}^{-1} \text{K}^{-1} \cdot 325 \text{K}}
            {9.8 \text{m s}^{-2}}\\
        H \approx \frac{\pi^2 \times \pi \times 10^2}{10} \text{m}\\
        H \approx \pi \times 10^2 \text{m} 
    \end{gather*}
\end{enumerate}
\end{document}