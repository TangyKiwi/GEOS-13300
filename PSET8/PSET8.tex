%! Author = Kevin Lin
%! Date = 5/23/2025

% Preamble
\documentclass[11pt,a4paper,margin=1in]{article}

% Packages
\usepackage{amsmath}
\usepackage{enumerate}
\usepackage{cancel}

\title{PSET 8}
\author{Kevin Lin}
\date{5/23/2025}

\begin{document}
\maketitle
\section{}
\begin{enumerate}[A.]
    \item 
    We know the Rossby number is given by $Ro = \frac{U}{fL}$, where we can
    approximate $U = 0.1$ m/s, the speed of the water draining out of the bucket,
    and for simplicity, also assume $L = 0.1$ m. Thus, we can estimate:
    \begin{gather*}
        Ro = \frac{U}{fL} \\
        Ro = \frac{0.1 \text{ m/s}}{2 \times 7.3 \times 10^{-5} \text{ s}^{-1} 
            \times \sin(0.1) \times 0.1 \text{ m}} \\
        Ro \approx \frac{10^{-1}}{2 \times 2\pi \times 10^{-5} \times 10^{-3}
            \times 10^{-1}} \\
        Ro \approx \frac{1}{10 \times 10^{-5} \times 10^{-3}} \approx 10^7 >> 1
    \end{gather*}
    Thus, we can conclude that the Coriolis force is negligible in this case.
    \item 
    In order for the Coriolis force to be negligible, we need $Ro << 1$. Thus:
    \begin{gather*}
        Ro = \frac{U}{fL} << 1 \\
        U << fL \\
        U << 2 \times 7.3 \times 10^{-5} \text{ s}^{-1} \times \sin(0.1) 
            \times 0.1 \text{ m} \\
        U << 10 \times 10^{-5} \times 10^{-3} \times 10^{-1} \text{ m/s} \\
        U << 10^{-8} \text{ m/s}
    \end{gather*}
    \item 
    No, it wouldn't matter. Although at the North Pole, the Coriolis force is 
    at its strongest, the minute amount of water and drainage speed would be
    still be negligible along with the Coriolis force, and $Ro$ would still be
    $>> 1$.
\end{enumerate}

\section{}
\begin{enumerate}[A.]
    \item 
    The air rotated clockwise in Cyclone Yasi.
    \item 
    From the first graph, we can see that Cyclone Yasi hit Australia at around 
    20$^{\circ}$S. We are given that $\rho = 1$ kg m$^{-3}$, and that a degree of
    latitude measures 111 km. From the second graph, we see that a degree of
    latitude measures a $\Delta p = 1004 - 1000 = 4$ hPa, thus, assuming 
    geostrophic balance, the wind speed is:
    \begin{gather*}
        \Delta p = V \rho f \Delta x \\
        4 \text{ hPa} = V \times 1 \text{ kg m}^{-3} \times 2 \times 7.3 \times
            10^{-5} \text{ s}^{-1} \times \sin(20^{\circ}) \times 111 \text{ km} \\
        4 \text{ hPa} \approx V \times \pi \text{ kg m}^{-2} \text{ s}^{-1} \\
        V = \frac{4 \text{ hPa}}{5.54 \text{ kg m}^{-2} \text{ s}^{-1}} \\
        4 \text{ hPa} = 400 \frac{\text{N}}{\text{m}^2} = 400 \frac{\text{kg}}
            {\text{ms}^2} \\
        V \approx  \frac{400 \text{kg}}{\text{ms}^2} \cdot \frac{1 \text{m}^2\text{s}^1}
            {\pi \text{kg}} \approx  100 \text{ m/s}
    \end{gather*}
    Compared to the speed of sound at sea level (340 m/s), this is a reasonable 
    answer for the wind speed of a cyclone.
    \item 
    Balancing the Coriolis and centrifugal forces against the pressure gradient
    force, and assuming $r = 111$ km, we have:
    \begin{gather*}
        fV + \frac{V^2}{r} = \frac{1}{\rho} \frac{\partial p}{\partial x}\\
        2 \times 7.3 \times 10^{-5} \text{ s}^{-1} \times \sin(20^{\circ}) \times
            V + \frac{V^2}{111 \text{ km}} = \frac{1}{1 \text{ kg m}^{-3}} 
            \frac{4 \text{ hPa}}{111 \text{ km}} \\
        \pi \times 10^{-5} \text{ s}^{-1} \times V + \frac{V^2}{10^5 \text{ m}} = 
            \frac{400 \text{ m}}{10^5 \text{ s}^2} \\
        \pi \text{ ms}^{-1} \times V + V^2 = 400 \frac{\text{m}^2}{\text{s}^2} \\
        V^2 + \pi V - 400 = 0\\
        V = \frac{-\pi \pm \sqrt{\pi^2 + 4 \times 400}}{2} \\
        V \approx \frac{-\pi + \sqrt{\pi^2 \times 10^2}}{2} \\
        V \approx \frac{-\pi + \pi \times 10}{2} \approx \pi^3 / 2 \approx \pi^2
            \approx 10 \text{ m/s}\\
    \end{gather*}
    This value considerably changes depending on our estimation of $r$.
\end{enumerate}
\end{document}