%! Author = Kevin Lin
%! Date = 5/2/2025

% Preamble
\documentclass[11pt,a4paper,margin=1in]{article}

% Packages
\usepackage{amsmath}
\usepackage{enumerate}
\usepackage{cancel}

\title{PSET 5}
\author{Kevin Lin}
\date{5/2/2025}

\begin{document}
\maketitle
\section{}
\begin{enumerate}[A.]
    \item 
    Given that the Moon has an albedo of 0.07, we can calculate the amount
    of absorbed flux as: $F_\text{abs} = (1 - A) \times 1360 \text{ Wm}^{-2} = (1
    - 0.07) \times 1360 \text{ Wm}^{-2} = 1264.8 \text{ Wm}^{-2}$. Using the 
    Stefan-Boltzmann law, we can find the temperature at the hottest point on the
    Moon to be:
    \begin{gather*}
        F_\text{abs} = \sigma T^4 \\
        T = \left( \frac{F_\text{abs}}{\sigma} \right)^{1/4} \\
        T = \left( \frac{1264.8 \text{ Wm}^{-2}}{5.67 \times 10^{-8} \text{ Wm}^{-2} 
            \text{K}^{-4}} \right)^{1/4} \\
        T = 386.465 \text{ K} = 235.967 \text{ }^{\circ}\text{F}
    \end{gather*}
    \item 
    Astronaut's wear space suits that are white to reflect sunlight and 
    minimize the amount of heat absorbed.
    \item 
    If it is noon and the equinox, the Sun is directly overhead the equator.
    Knowing Chicago is $42^{\circ}$N, then the amount of flux striking is 
    $F = \cos(42^{\circ}) \times 1360 \text{ Wm}^{-2} = 1010.68 \text{ Wm}^{-2}$. 
    For noon on the winter solstice, given that Earth's obliquity is $23^{\circ}$,
    the amount of flux striking is $F = \cos(42^{\circ} + 23^{\circ}) \times 1360
    = 574.76 \text{ Wm}^{-2}$. For the summer solstice, the flux striking is $F 
    = \cos(42^{\circ} - 23^{\circ}) \times 1360 = 1285.91 \text{ Wm}^{-2}$. 
    \item 
    We know that the sun strikes a circular area of the Earth, thus the total power
    of sunlight striking the Earth is:
    \begin{gather*}
        W_\text{total} = \pi R^2 \times F \\
        W_\text{total} = \pi (6.4 \times 10^6 \text { m})^2 \times 1360 \text{ Wm}^{-2}
    \end{gather*}
    Averaged over the Earth's entire surface, the flux is:
    \begin{gather*}
        F_\text{avg} = \frac{W_\text{total}}{4 \pi R^2} \\
        F_\text{avg} = \frac{\cancel{\pi (6.4 \times 10^6 \text { m})^2} \times 1360 
            \text{ Wm}^{-2}}{4 \times \cancel{\pi (6.4 \times 10^6 \text { m})^2}} \\
        F_\text{avg} = \frac{1360}{4} \text{ Wm}^{-2} = 340 \text{ Wm}^{-2}
    \end{gather*}
    \item 
    If Earth had no greenhouse gases, the surface temperature would be:
    \begin{gather*}
        F = \sigma T^4 \\
        T = \left( \frac{F}{\sigma} \right)^{1/4} \\
        T = \left( \frac{340 \text{ Wm}^{-2}}{5.67 \times 10^{-8} \text{ Wm}^{-2} 
            \text{K}^{-4}} \right)^{1/4} \\
        T = 278.275 \text{ K}
    \end{gather*}
    If the planet had an average albedo of 0.3, then the surface temperature would
    be:
    \begin{gather*}
        (1 - A) \times F = \sigma T^4 \\
        (1 - 0.3) \times 340 \text{ Wm}^{-2} = \sigma T^4 \\
        0.7 \times 340 \text{ Wm}^{-2} = \sigma T^4 \\
        T = \left( \frac{0.7 \times 340 \text{ Wm}^{-2}}{5.67 \times 10^{-8} 
            \text{ Wm}^{-2} \text{K}^{-4}} \right)^{1/4} \\
        T = 254.536 \text{ K}
    \end{gather*}
    \item
    From class, we calculated the equation for the surface temperature of the Earth
    with a single-layer atmosphere with emissivity $\epsilon$ to be:
    \begin{gather*}
        T_s = \left( \frac{S(1 - A)}{\sigma(4 - 2\epsilon)} \right)^{1/4}
    \end{gather*}
    Thus, for an albedo of 0.3 and surface temperature of 278 K, the emissivity
    is:
    \begin{gather*}
        287 K = \left( \frac{1360 \text{ Wm}^{-2} (1 - 0.3)}{5.67 \times 10^{-8} 
            \text{ Wm}^{-2}\text{K}^{-4} (4 - 2\epsilon)} \right)^{1/4} \\
        \epsilon = 0.76
    \end{gather*}
    \item 
    Using Kirchhoff's law, the longwave absorptivity of this single-layer atmosphere
    is equal to the emissivity, thus $\alpha = \epsilon = 0.76$. Assuming no scatter,
    the longwave transmissivity is $T = 1 - \alpha = 0.24$. Using Beer's law, the 
    optical thickness of the atmosphere is simply:
    \begin{gather*}
        \tau = -\ln T\\
        \tau = -\ln(0.24) \\
        \tau = 1.427
    \end{gather*}
\end{enumerate}
\end{document}